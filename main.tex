\documentclass{article}
\usepackage[utf8]{inputenc}
\usepackage[table]{xcolor}
\usepackage[rightcaption]{sidecap}
\usepackage{amsmath}
\usepackage{graphicx}
\usepackage[colorinlistoftodos]{todonotes}
\usepackage[colorlinks=true, allcolors=blue]{hyperref}
\usepackage[toc,page]{appendix}
\usepackage[utf8]{inputenc}
\bibliographystyle{stylename}
\bibliography{bibfile}
\usepackage{graphicx} %package to manage images
\usepackage{xcolor}
\newcommand\dubtes[1]{\textcolor{red}{#1}}
\graphicspath{ {images/} }
\title{Pràctica 0}
\usepackage{listings}
\usepackage{subfig}
\usepackage{wrapfig}
\begin{document}


\title{Guia orientativa per el FFL (EPSEM)}

\maketitle{\textbf{Prefaci}}
\hfill
\textit{Això és un document orientatiu que sintetitza l'explicació que ens ofereixen al Webinar per jutges\cite{Webinar} juntament amb una petita explicació que és la First Lego League (FLL)\cite{FFL}.}
\hfill 

\section{Introducció}
\hfill

Competició orientada cap a gent d’entre 8 i 16 anys, on els ítems principals de la competició seran programació, mecànica, investigació, creativitat, treball en equip i innovació/comunicació. 

\hfill \break\hfill \break
És una experiència força guiada on cada activitat i els sets ja tenen material didàctic preparat, en cada programa es desenvolupa un projecte innovador per resoldre un problema real relacionat amb la temàtica de la competició.

\section{Temàtica Explore}
No competitiva, orientada als infants. Els equips expliquen i ensenyen el treball portat durant la temporada. Treballen en aquests tres àmbits:

\begin{itemize}
  \item Valors FFL: Quan els revisors parlin amb aquests equips, els hi demanaran com han treballat, com s’han organitzat, per a què els hi ha agradat, una conversa amb l’equip per determinar que compleixin els valors següents: Descobriment, inclusió, innovació, impacte, col·laboració i presentació
  \item Poster que resumeix el treball de la temporada i els ítems esmenats anteriorment. El pòster exposa de manera senzilla el treball realitzat
  \item Maqueta LEGO: El muntatge amb el model de la temporada, aquesta maqueta ha de mostrar:
  \begin{itemize}
    \item Estació d'exploració
    \item Ha de estar motoritzada
    \item Ha d'incloure tapete 
    \item Ha de utilitzar els models del set del any actual
  \end{itemize}
\end{itemize}



\section{Presentació Temàtica Explore}

Els equips han de demostrar els valors FFL, mostrar el seu pòster i la seva maqueta.
\break
\break
\begin{tabular}{ |c|c| } 
 \hline
 \textbf{Categoria} & \textbf{Temps (minuts)}\\ 
 Benvinguda al equip  & 2'  \\ 
 Preguntes sobre valors & 2'  \\ 
 Mostra i preguntes sobre el pòster & 5' \\ 
 Mostra i preguntes sobre la maqueta & 5'  \\ 
 Acomiadament & 1'  \\ 
 \hline
\end{tabular}


\section{Temàtica Challenge}

Competitiva, orientada als més grans. Dividida en 4 parts (\textbf{Innovació, Disseny, Valors i Joc} ), totes d'igual importància, cadascuna consta d’una presentació de 5 minuts.
\hfill \break\hfill \break
Com a primera part tenim \textbf{Innovació del projecte} (Investigació sobre la temàtica de l'any). Els equips presentaran el projecte davant del jurat durant 5 minuts, consisteix en les següents fases:

\begin{itemize}
    \item Identificar un problema real relacionat amb la temàtica del problema
    \item Investigar el problema i les solucions
    \item Dissenyar una solució
    \item Compartir la solució i iterar sobre ella
    \item Comunicar la solució en si
\end{itemize}

Per al que fa el \textbf{Disseny del robot}, s'ha d'explicar per què el robot és com és, els membres de l'equip hauran de realitzar una presentació del disseny i explorar l'enfoc de programació, això inclou: Programació en si, disseny, mecànica i sensorització, estratègia de missions i comunicació.
\hfill \break\hfill \break
Llavors tenim els \textbf{Valors fonamentals}, aquí valorarem els valors de l'equip. S'analitzarà com l'equip integra els valors FIRST en els diferents àmbits del problema i la seva experiència del dia a dia, no tenim rúbrica de valoració específica, els valors s'avaluen de manera transversal durant tot moment, per exemple:

\begin{itemize}
    \item Inclusió i treball en equip
    \item Descobriment i inovació
    \item Diversió
\end{itemize}

Finalment tenim el \textbf{Joc del robot} és l'activitat que porta a terme el robot dissenyat.


\section{Desafiament}

Aquesta temporada el desafiament consta en els problemes que s'enfronten les persones que exploren els oceans, durant el webinar fa èmfasi en què \textbf{no necessàriament} ha de ser en l'àmbit professional, qualsevol problema enfocat des de l'angle correcte és vàlid mentre la temàtica sigui exploració de l'oceà.
\hfill \break\hfill \break
Cada equip ja d’identificar una dificultat i buscar una solució, per defecte ofereixen 4 professions d’exemple i els problemes que es troben, alguns equips també poden investigar altres activitats relacionades amb la temàtica. Com a material tenim Quadern d'enginyeria i guia de reunions de l'equip, el material és accessible per als entrenadors 


\section{Estructura de presentació}

Cada equip te assignat un tribunal, una sala i una hora determinada. Els equips dialogen amb el tribunal i arranca una sessió continua de 30’.

\hfill \break\hfill \break
\begin{tabular}{ |c|c| } 

 \hline

 \textbf{Categoria} & \textbf{Temps (minuts)} \\ 

 Benvinguda a l'equip  & 2'  \\ 

 Projecte innovació & 5' de presentació + 5' de preguntes   \\ 

 Disseny del robot & 5' de presentació + 5' de preguntes  \\ 

 Valoració dels jutges & 10'  \\ 

 Acomiadament & 1'   \\ 

 \hline

\end{tabular}

\hfill \break\hfill \break
Per a la presentació qualsevol metodologia és vàlida (powerpoint, etc). Per poder fer la presentació els primers 2' l’equip prepara el material, el jurat es presenta i sobretot és tot molt distes per treure'ls-hi els nervis. Important en la valoració: Això és un diàleg amb els jutges, no és un examen ni tesis doctoral, un diàleg seriós, però informal, venen a passar-s’ho bé. 
\hfill \break\hfill \break
Ofereixen ja unes rúbriques d’exemple, estan en l’àrea privada de ingenier@soy
\hfill \break\hfill \break
Tant projecte d’innovació com el disseny tenen rúbrica normal on s'assignen puntuacions del 1 al 4 (bàsic -> superat). La de valoració és més transversal.

\section{Premis}


A part dels premis normals, també parla d’un premi al Entrenador/a, aquesta persona ha d’estar nominada i aquest full ens ho entregaran al mateix dia del tribunal, No se com es determina qui es millor entrenador. 
\hfill \break\hfill \break
Hi ha tambe el premi al companyerisme extraordinari, només per events amb més de 16 equips, els grups es nominen entre si i reconeixen la feina dels altres equips en funció de feina, valors, colaboració, etc. Es nominen entre ells amb un full i aquest esta disponible durant el dia de l’event en si. Entenc que abans de la ceremonia de premis tambe determinem quin equip guanya en funció de les nominacións que hagui rebut. 

Un altre premi es el NIA (national Innovation Award), una nominació que es fa a cada seu local, a la final nacional es determina el guanyador d’entre tots els nominats.
\hfill \break\hfill \break
En funció del event tenim una bona varietat de premis, alguns son condicionals al numero d’equips participants:

\begin{itemize}
    \item Premi Ingeneria Soy (Equip guanyador)
    \item Premi als valors FIRST
    \item Premi al Projecte d’Innovació
    \item Premi al Diseny del Robot
    \item Premi al Comportament del Robot
    \item Premi a les Joves promeses
    \item Premi a l’Excelencia a l’enyengeria \#*
    \item Premi a L’emprenadoria \#*
    \item Premi al companyerisme extraordinari \#*
    \item Premi al Entrenador/a
\end{itemize}

Respecte als premis hi ha unes determinades normes, un equip \textbf{NO} pot repetir premis excepte en un cas com aquest:

\begin{itemize}
    \item Premi de comportament del robot \textit{premi generic}
    \item Premi al Entrenador/a \textit{és a titol personal, no d'equip}
    \item Premi al companyerisme extraordinari \textit{determinat per els altres equips}
\end{itemize}


\section{QA}

Recopilo algunes de les preguntes que he trobat més interessants + observacions:
\hfill \break\hfill \break
- \textbf{Si no hi ha nominacions al companyerisme extraordinari com es procedeix?} És fàcil que hi hagui nominacions, si no n'hi ha cap, més detalls a la sessió de formació específica de jutges (m’he quedat igual)
\hfill \break\hfill \break
- \textbf{Si que es pot tenir el Innovation award i un altre premi?} Si, perquè no és un premi en si, es una nominació
\hfill \break\hfill \break
- \textbf{S’ha d’entregar un document de projecte o venen directament amb el seu powerpoint?} L’equip té 5 minuts per exposar el que volen, no han d’enviar res i poden penjar el que vulguin al eventhub, l'important és la presentació en si.
\hfill \break\hfill \break
- \textbf{En tots els events s’utilitza el eventHub?} Si, es obligatori
\hfill \break\hfill \break
- S’intenta per tots els mitjans que hi hagui públic a les presentacions
\hfill \break\hfill \break
- Qualsevol temàtica relacionada amb el problema es valida si s’enfoca des de l’angle correcte
\hfill \break\hfill \break
- \textbf{S’ha d’arribar a implementar el prototip? És suficient amb l’explicació del problema?} Depen de la complexitat del problema, però si que han de tenir la capacitat de fer una maqueta (encara que no sigui funcional) per explicar millor la solució.
\hfill \break\hfill \break
- \textbf{Que passa si porten altres materials que no siguin LEGO?} Ha de ser LEGO
\hfill \break\hfill \break
- S’ha de buscar un problema real però focalitzat en l’exploració dels oceans, tant recreatiu com professional, ha d’acabar sent d’ajuda a la temàtica plantejada.

\hfill \break\hfill \break
\textbf{P: Ha d’incloure tapete es refereixen a la ”manta” sobre la que el Lego treballa}
R: Sí, sempre que escoltin "tapete" es refereix a aquest mapa que es posa a sobre de les taules de competició.
\hfill \break\hfill \break
\textbf{P: Falta txitxa, però entenc que la rúbrica serà un ”fa el que ha de fer"}
R: Al final la competició que veuen els jutges es basa en aquests quatre punts, no hi ha gaire més. La rúbrica és per als jutges a l'hora de fer la valoració de cada equip, si no m'equivoco hi ha una rúbrica per cada punt (una per la presentació del projecte d'innovació, una per la part del disseny del robot amb la seva programació...). Tindreu totes les rubriques amb anterioritat de forma que els hi podreu donar una ullada abans del dia.
\hfill \break\hfill \break
\textbf{P: Tenim accés al Quadern d'enginyeria i a la guia de reunions de l'equip?}
R: Aquests dos documents estan fets per ajudar els equips a tenir una organització i un seguiment del seu treball, és una cosa completament opcional i no és necessari que ho portin a la presentació, sempre podeu preguntar a l'equip directament si ho han fet servir o com han fet per organitzar-se i guardar la informació
\hfill \break\hfill \break
\textbf{P: Les rúbriques s'omplen en físic o es fa des del portal de l'event (EventHUB)?}
R: Es pot fer de les dues formes, podem posar un ordinador a cada sala on els jurats podran omplir i penjar al portal les rubriques directament, o podem imprimir rubriques i després tenir a un parell de persones designades a anar a buscar-les després de cada presentació i passar-les al portal, quina de les dues formes fem depèn de si tenim suficients voluntaris per fer-ho a paper i de les preferències dels jurats.
\hfill \break\hfill \break
\textbf{P: Com es determina el premi a millor entrenador?}
R: Es dona d'acord amb el que veiem nosaltres durant el dia, si veieu a algun entrenador que anima molt al seu equip i ho fa bé, després durant la deliberació ho dieu i entre tots es parla a veure quin entrenador creiem que es mereix aquest premi.
\hfill \break\hfill \break
\textbf{P: Quins premis donem?}
R: Donem els premis:
     -  INGENIERA SOY al guanyador de l'event
     -  VALORS
     -  PROJECTE D'INNOVACIÓ
     -  DISSENY DEL ROBOT
     -  COMPORTAMENT DEL ROBOT
     -  JOVES PROMESES
     -  EXCEDÈNCIA DE L'ENGINYERIA
     -  ENTRENADOR
     \hfill \break\hfill \break
\textbf{P: Donem la nominació NIA?}
R: Si, l'EPSEM és una seu i fa una nominació per a aquest premi.
\hfill \break\hfill \break
\textbf{P: Accés a EventHUB}
R: Si, tindreu accés, ja us passarem tan bon punt quan puguem
\hfill \break\hfill \break
\begin{thebibliography}{9}

\bibitem{Webinar}
¡Descubre FIRST LEGO League Challenge!

https://www.youtube.com/watch?v=AaQVzdTOs-k
\bibitem{FFL}
Seminario Proyecto Innovacion, FIRST LEGO League Spain, Temporada 2024/25  
https://www.youtube.com/watch?v=6mfT9iQOApE

\end{thebibliography}

\end{document}

